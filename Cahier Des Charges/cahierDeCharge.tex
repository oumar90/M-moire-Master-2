\documentclass[12pt,a4paper]{article}
\usepackage[utf8]{inputenc}
\usepackage[french]{babel}
\usepackage[T1]{fontenc}
\usepackage{amsmath}
\usepackage{amsfonts}
\usepackage{amssymb}
\usepackage{graphicx}
\usepackage[left=2cm,right=2cm,top=2cm,bottom=2cm]{geometry}
\author{Oumar Djimé Ratou}

\title{Cahier de charge}
\title{Développement d'un module de signature électronique et de PKI}
\author{Oumar Djimé Ratou}

\begin{document}
\maketitle


\newpage
\tableofcontents
\newpage





\section{Description et compréhension du projet}
Le module de signature électronique et de PKI(Public Key Infrastructure) ou infrastructure à clé publique est une service permettant de créer des signatures numériques et des infrastructures à clé publique qui permettra à d'autre développeurs d'intégré facilement dans leurs plate-forme de même technologie afin que les utilisateurs finaux puissent l'utiliser aisément.

\section{Étude de la faisabilité technique}
	\subsection{Contexte et problématique}
	\subsection{Objectifs}
	\subsubsection{Objectif global}
	\subsubsection{Objectif spécifique}


\section{Description des besoins}
	\subsection{Spécifications fonctionnelles}
	\subsection{Spécifications non-fonctionnelles}

\section{Enveloppe budgétaire}

\section{Délai}


\end{document}